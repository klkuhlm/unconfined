\NeedsTeXFormat{LaTeX2e} [1994/06/01]
\documentclass[12pt,letterpaper]{article}
\usepackage[mathlines,pagewise]{lineno}
\usepackage{graphicx}
\usepackage{amsmath}

\renewcommand{\baselinestretch}{1.25}
\renewcommand{\arraystretch}{1}

\setlength{\oddsidemargin}{0.0in} \setlength{\hoffset}{0in}
\setlength{\textwidth}{6.5in}
\setlength{\topmargin}{0cm} \setlength{\voffset}{0.0in}
\setlength{\textheight}{8.25in}

% Needed to get lineno to work nicely with amsmath
\newcommand*\patchAmsMathEnvironmentForLineno[1]{
 \expandafter\let\csname old#1\expandafter\endcsname\csname #1\endcsname
 \expandafter\let\csname oldend#1\expandafter\endcsname\csname end#1\endcsname
 \renewenvironment{#1}
 {\linenomath\csname old#1\endcsname}
 {\csname oldend#1\endcsname\endlinenomath}}
\newcommand*\patchBothAmsMathEnvironmentsForLineno[1]{
 \patchAmsMathEnvironmentForLineno{#1}
 \patchAmsMathEnvironmentForLineno{#1*}}
\AtBeginDocument{
\patchBothAmsMathEnvironmentsForLineno{equation}
\patchBothAmsMathEnvironmentsForLineno{align}}

\title{Alternate Formulation for Mishra-Neuman Solution}
\begin{document}
\maketitle
\linenumbers
\section{Unsaturated Zone Governing Equations}
The governing equation in the unsaturated zone, as used by \cite{mishra10} and \cite{tartakovsky07}, is
\begin{equation}
  \label{eq:unsatDim}
  K_r k_0(\psi) \frac{1}{r} \frac{\partial}{\partial r} \left( r\frac{\partial \sigma}{\partial r} \right) + K_z \frac{\partial}{\partial z} \left( k_0(z) \frac{\partial \sigma}{\partial z}\right) = C(\psi) \frac{\partial \sigma}{\partial t}
\end{equation}
where $K_r$ and $K_z$ are the radial and vertical saturated hydraulic conductivities [L T$^{-1}$], $k_0$ is the dimensionless isotropic relative hydraulic conductivity ($0 < k_0 \le 1$), $\sigma = h_0 - h = b +\psi_a - h$ is drawdown in the vadose zone [L], $C(\psi)=\frac{d\eta}{d \psi}$ is the dimensionless specific moisture capacity, and $\theta$ is dimensionless volumetric water content.  The unsaturated zone begins at the top of the aquifer, and is defined as $b \le z \le b+L$.  Further, the unsaturated parameters are assumed to only be functions of $z$, $k_0(z)=k_0(\theta_0)$ and $C_0(z)=C_0(\theta_0)$.  The initial and boundary conditions for the unsaturated zone are  
\begin{equation}\nonumber
\sigma(r,z,0) = 0
\end{equation}
\begin{equation}\nonumber
\sigma(\infty,z,t)=0
\end{equation}
\begin{equation}\nonumber
\frac{\partial \sigma}{\partial z}=0 \qquad z=b+L
\end{equation} and 
\begin{equation}\nonumber
\lim_{r \rightarrow 0} r \frac{\partial \sigma}{\partial r} = 0 \qquad b\le z \le b+L.
\end{equation}

Drawdown in the aquifer, $s$, and the vadose zone $\sigma$ are connected using compatibility conditions at the water table ($z=b$);
\begin{equation}\nonumber
s=\sigma \qquad z=b
\end{equation}
\begin{equation}\nonumber
\frac{\partial s}{\partial z}=\frac{\partial \sigma}{\partial z} \qquad z=b.
\end{equation} 

The moisture retention curve in the vadose zone is represented as an exponential function
\begin{equation}\nonumber
S_e = \frac{\theta(\psi) - \theta_r}{S_y} = e^{a_c \left( \psi - \psi_a \right)} \qquad a_c \ge 0 
\end{equation}
where $S_e$ is dimensionless effective saturation, $\theta_r$ is dimensionless residual water content, and $S_y=\theta_s - \theta_r$ is dimensionless drainable porosity (specific yield).  The Gardner model for relative hydraulic conductivity is used; the relative hydraulic conductivity is
\begin{equation}
  \nonumber
  k(\psi) = \left  \{ 
    \begin{array}{c c}
      e^{a_k \left( \psi - \psi_k\right)} & \psi \le \psi_k \\
      1 & \psi < \psi_k\\
    \end{array} 
\right . \qquad a_k \ge 0.
\end{equation}
The four-parameter exponential model for hydraulic conductivity used in the vadose zone is 
\begin{equation}
  \label{eq:Gardner}
 k_0(z)=e^{a_k\left( b_1 + b - z\right)} \qquad b_1=\psi_a-\psi_k
\end{equation}
and the moisture capacity is
\begin{equation}
  \label{eq:mrc}
C_0(z) = S_y a_c e^{a_c \left( b-z\right)}.
\end{equation}

The governing equation  must be non-dimensionalized first; \eqref{eq:unsatDim} can be written in dimensionless form as
\begin{equation}
  \label{eq:unsatDimless}
  k_0(\psi_D) \frac{1}{r_D} \frac{\partial}{\partial r_D} \left( r_D\frac{\partial \sigma_D}{\partial r_D} \right) + \kappa \frac{\partial}{\partial z_D} \left( k_0(z_D) \frac{\partial \sigma_D}{\partial z_D}\right) = \gamma C_0(z_D) \frac{\partial \sigma_D}{\partial t_D}
\end{equation}
where $r_D=r/b$ is the dimensionless radial coordinate, $z_D=z/b$ is the dimensionless vertical coordinate, $\kappa=K_z/K_r$ is the anisotropy ratio, $\sigma_D = \sigma/H_c$ is dimensionless vadose zone drawdown, $H_C$ is a characteristic head, $t_D = t K_r / (b^2 S_S)$ is dimensionless time, and $\gamma = S_y a_c/S_S$ is the dimensionless storage ratio.  The hydraulic conductivity constituative model is non-dimensionalized as
\begin{equation}
  \label{eq:GardnerDimless}
 k_0(z_D)=e^{a_{kD} \left( b_{1D} + 1 - z_D \right)},
\end{equation}
where $a_{kD} = a_k b$ and $b_{1D}=b_1/b$, and the moisture capacity model is non-dimensionalized as
\begin{equation}
  \label{eq:mrcDimless}
C_0(z_D) = e^{a_{cD} \left( 1-z_D\right)},
\end{equation}
where $a_{cD} = a_c b$.

\section{Unsaturated Zone Solution}
The dimensionless Laplace transformation (overbar and $p$) of (\ref{eq:unsatDimless}) results in
\begin{equation}
  \label{eq:unsatLap}
   k_0(z_D) \frac{1}{r_D} \frac{\partial}{\partial r_D} \left( r_D\frac{\partial \bar{\sigma}_D}{\partial r_D} \right) +\kappa \frac{\partial}{\partial z_D} \left( k_0(z_D) \frac{\partial \bar{\sigma}_D}{\partial z_D}\right) = p \gamma C_0(z_D)  \bar{\sigma}_D
\end{equation}
where , with the associated transformed dimensionless boundary and initial conditions of
\begin{equation}\nonumber
 \bar{\sigma}_D(\infty,z_D,p) = 0
\end{equation}
\begin{equation}\nonumber
 \frac{\partial \bar{\sigma}_D}{\partial z_D}=0 \qquad z_D=1+L_D
\end{equation}
 \begin{equation}\nonumber
\lim_{r_D \rightarrow 0} r_D \frac{\partial \bar{\sigma}_D}{\partial r_D} = 0 \qquad 1\le z_D \le 1+L_D
\end{equation}
The dimensionless Hankel transformation (superscript $\ast$ and $a$) of (\ref{eq:unsatLap}) and substitution of the constitutive models (\ref{eq:Gardner}) and (\ref{eq:mrc}) results in the ordinary differential equation
\begin{equation}
  \label{eq:unsatHank}
   -a^2 \bar{\sigma}^{\ast} + \kappa a_k \frac{\mathrm{d}^2 \bar{\sigma}^{\ast}}{\mathrm{d} z^2}  = p e^{-a_k b_1} \frac{S_y a_c} {K_r}  e^{\left(a_k -a_c \right)\left( b - z\right)} \bar{\sigma}^{\ast} 
\end{equation}
and the no-flow condition at the top of the vadose zone, 
\begin{equation}\nonumber
 \frac{\mathrm{d}\bar{\sigma}^{\ast}}{\mathrm{d}z}=0 \qquad z=b+L 
\end{equation}
Equation \ref{eq:unsatHank} can be rearranged as
\begin{equation}
  \label{eq:mn-d5}
  \frac{\mathrm{d}^2 \bar{\sigma}^{\ast}}{\mathrm{d}z^2} - a_k \frac{\mathrm{d} \bar{\sigma}^{\ast}}{\mathrm{d}z} - \left( B e^{\lambda (z-b)} + C\right) \bar{\sigma}^{\ast}=0
\end{equation}
where $\lambda = a_k-a_c$, $C=\frac{a^2}{\kappa}$ and
\begin{equation}\nonumber
B = p e^{-a_k b_1} \frac{S_y a_c} {K_r} 
\end{equation}
which was solved in \cite{mishra10} using a general solution from a
reference book on differential equations.  The solution given in terms
of two types of Bessel functions of complex argument and non-integer
order is non-trivial to evaluate numerically. 

To combine with existing solutions, \eqref{eq:mn-d5} must be made
non-dimensional.  One non-dimensionalization is  
\begin{equation}
  \label{eq:mn-d5-dimless}
  \frac{\mathrm{d}^2 \bar{\sigma}_D^{\ast}}{\mathrm{d}z_D^2} - \beta_3 \frac{\mathrm{d} \bar{\sigma}_D^{\ast}}{\mathrm{d}z_D} - \left( B_1 e^{\beta_1 (1-z_D)} + B_2\right) \bar{\sigma}_D^{\ast}=0
\end{equation}
where $\sigma_D$ and $z_D$ are dimensionless drawdown and vertical
coordinate, while the following dimensionless quantities are defined:
$B_1 = e^{-\beta_2}  p \beta_0/\kappa $, 
$B_2=a^2/\kappa$, $\beta_0=a_c S_y/S_s$, $\beta_1=b(a_c - a_k)$,
$\beta_2=a_k b_1$, $\beta_3 = a_k b$.

\subsection{Modified Solution Procedure} 
By further modifying the $z_D$-coordinate and performing an exponential substitution, (\ref{eq:mn-d5-dimless}) can be simplified significantly and a solution can be found by integration.  Using the new dimensionless vertical coordinate 
\begin{equation}\nonumber
 \zeta=\beta_3 z_D \qquad z_D=\frac{\zeta}{\beta_3} 
\end{equation}
the transformed differential equation representing drawdown in the vadose zone (\ref{eq:mn-d5}) becomes
\begin{equation}
  \label{eq:nondimODE}
  \beta_3^2 \frac{\mathrm{d}^2
    \bar{\sigma}_D^{\ast}}{\mathrm{d}\zeta^2} - \beta_3^2
  \frac{\mathrm{d} \bar{\sigma}_D^{\ast}}{\mathrm{d}\zeta} - \left(
    B_1 e^{\beta_1 (1-\zeta/\beta_3)} + B_2\right) \bar{\sigma}_D^{\ast}=0
\end{equation}
dividing through by $\beta_3^2$ and performing the substitution
$\bar{\sigma}_D^{\ast}(\zeta)=e^{u(\zeta)}$ which implies

\begin{equation}\nonumber
\frac{\mathrm{d}\bar{\sigma}_D^{\ast}}{\mathrm{d}\zeta} =
\frac{\mathrm{d}u}{\mathrm{d}\zeta}e^u
\end{equation} 
and
\begin{equation}\nonumber
\frac{\mathrm{d}^2\bar{\sigma}_D^{\ast}}{\mathrm{d}\zeta^2} =
\frac{\mathrm{d}^2u}{\mathrm{d}\zeta^2}e^u +
\frac{\mathrm{d}u}{\mathrm{d}\zeta}e^u
\end{equation}
leads to
\begin{equation}\nonumber
 \left[ \frac{\mathrm{d}^2u}{\mathrm{d}\zeta^2}e^u +
  \frac{\mathrm{d}u}{\mathrm{d}\zeta}e^u \right]-  \left[
  \frac{\mathrm{d}u}{\mathrm{d}\zeta}e^u \right]- \frac{1}{\beta_3^2}\left( B_1 e^{\beta_1 (1-\zeta/\beta_3)} + B_2\right) e^{u}=0
\end{equation}
multiplying by $e^{-u}$ and simplifying leads to
\begin{equation}
  \label{eq:expsubODE}
   \frac{\mathrm{d}^2 u}{\mathrm{d}\zeta^2} = \left( \frac{B_1}{\beta_3^2} e^{\beta_1 (1-\zeta/\beta_3)} + \frac{B_2}{\beta_3^2}\right) 
\end{equation}

This is an integrable equation; one indefinite integration produces
\begin{equation}
  \label{eq:du}
  \frac{\mathrm{d} u}{\mathrm{d}\zeta} = - \frac{B_1}{\beta_1 \beta_3} e^{\beta_1(1-\zeta/\beta_3)} + \frac{B_2}{\beta_3^2}\zeta + \alpha
\end{equation}
where $\alpha$ is a constant of integration. A second indefinite integration produces
\begin{equation}\nonumber
 u =  \frac{B_1}{\beta_1^2} e^{\beta_1 (1-\zeta/\beta_3)} + \frac{B_2}{2\beta_3^2}\zeta^2 + \alpha\zeta + \gamma  
\end{equation}
where $\gamma$ is another constant of integration.  

Applying the exponential transformation to the boundary condition at the top of the vadose zone results in
\begin{equation}\nonumber
\frac{\mathrm{d}u}{\mathrm{d}\zeta} e^u=0 \qquad \zeta= \left(1+ L_D \right) \beta_3
\end{equation}
which is used with \eqref{eq:du} to determine one of the constants
\begin{equation}\nonumber
\alpha= \frac{B_1}{\beta_1 \beta_3} e^{\beta_1 L_D} -
\frac{B_2}{\beta_3} \left( 1+L_D \right) 
\end{equation}

The solution for the drawdown in the vadose zone is then
\begin{equation}
  \nonumber
  \bar{\sigma}_D^{\ast} (z_D) = \tilde{\gamma}\exp\left\{
    \frac{B_1}{\beta_1^2} e^{\beta_1 (1-z_D)} + \frac{B_2}{2}z_D^2
    +\left[ \frac{B_1}{\beta_1} e^{\beta_1 L_D}  - B_2 \left( 1+L_D\right) \right]z_D  \right\}
\end{equation}
where $\tilde{\gamma}=e^\gamma$ is a modified integration constant. This can be re-grouped as
\begin{equation}
  \label{eq:sigma}
  \bar{\sigma}_D^{\ast} (z_D) = \tilde{\gamma}\exp\left\{
    \frac{B_1}{\beta_1^2} \left[ e^{\beta_1 (1-z_D)} + z_D\beta_1 e^{\beta_1
        L_D} \right] + B_2 \left[ \frac{z_D^2}{2}
     - \left( 1+L_D\right)z_D  \right] \right\}
\end{equation}
while $\tilde{\gamma}$ will be determined from the matching conditions at the water table.
\section{Saturated Zone Unconfined Solution}
Following \cite{mishra10}, we decompose the saturated zone solution into two solutions: 1) the Hantush solution $s_H$ for a confined aquifer and a partially penetrating well; 2) an unconfined solution $s_U$ with no well but continuity conditions with the vadose zone solution just derived.  The solution for $s_H$ is given elsewhere, either in terms of a finite cosine transform, or a multi-layer fully penetrating solution.  

The governing ordinary differential equation in Laplace-Hankel
transform space for the saturated unconfined $s_{UD}$ is
\begin{equation}
  \label{eq:aquifer}
  \frac{\mathrm{d}^2 \bar{s}_{UD}^{\ast}}{\mathrm{d} z_D^2} - \left( \frac{p + a^2}{\kappa} \right)\bar{s}_U^{\ast} = 0 \qquad 0 \le z_D \le 1
\end{equation}
with the no-flow boundary condition at the base of the aquifer
\begin{equation}\nonumber
 \frac{\mathrm{d} \bar{s}_{UD}^{\ast}}{\mathrm{d} z_D} = 0 \qquad z_D=0 
\end{equation}
and the following continuity equations at the water table
\begin{equation}
  \label{eq:headCont}
  \bar{s}_{UD}^{\ast} + \bar{s}_{HD}^{\ast} = \bar{\sigma}_D^{\ast} \qquad z_D=1
\end{equation}
\begin{equation}
  \label{eq:fluxCont}
  \frac{\mathrm{d} \bar{s}_{UD}^{\ast}}{\mathrm{d} z_D}  = \frac{\mathrm{d}
    \bar{\sigma}_D^{\ast}}{\mathrm{d} z_D} \qquad \rightarrow \qquad \frac{\mathrm{d} \bar{s}_{UD}^{\ast}}{\mathrm{d} z_D}  = \frac{\mathrm{d}u}{\mathrm{d} \zeta} e^u \qquad z_D=1
  \qquad 
\end{equation}
note the vertical flux of $\bar{s}_{HD}^{\ast}$ is defined as zero at
the top and bottom of the aquifer, so it is not included in the
continuity condition \eqref{eq:fluxCont}.

The general solution to \eqref{eq:aquifer} is 
\begin{equation}
  \label{eq:su}
  \bar{s}_{UD}^{\ast} = A_1 \cosh(\eta z_D) + A_2 \sinh(\eta z_D)
\end{equation}
where $A_i$ are constants to determine and $\eta^2 = \frac{p + a^2}{\kappa}$.  The no-flow boundary condition at the bottom of the aquifer forces $A_2=0$.  

\section{New Solution for Saturated Zone}
Substituting \eqref{eq:sigma} and \eqref{eq:su} into the compatibility
equations \eqref{eq:headCont} and \eqref{eq:fluxCont} gives
\begin{equation}\nonumber
  A_1 \cosh(\eta) + \bar{s}_{HD}^{\ast}(z_D=1) = \tilde{\gamma}e^{\omega_1}
\end{equation}
\begin{equation}\nonumber
 A_1 \eta  \sinh(\eta)  = \tilde{\gamma} \omega_2 e^{\omega_1}
\end{equation}
where 
\begin{equation}\nonumber
 \omega_1 = \frac{B_1}{\beta_1^2} \left( 1 + \beta_1 e^{\beta_1
        L_D} \right) - B_2 \left( \frac{1}{2}
     +L_D  \right) 
\end{equation}
\begin{equation}
  \label{eq:omega2}
  \omega_2 = \frac{B_1}{\beta_1} \left( e^{\beta_1 L_D} -1 \right) - B_2 L_D 
\end{equation}
are $u(z_D=1)$ and $\left. \frac{\mathrm{d} u}{\mathrm{d}z_D} \right|_{z_D=1}$
respecively (without the $\tilde\gamma$ factor).
Divide the second equation by $\omega_2$ and set the left-hand-sides of the two resulting equations equal to one another.  Solving for $A_1$ leads to
\begin{equation}\nonumber
A_1 \cosh(\eta) + \bar{s}_{HD}^{\ast}(z_D=1) = \frac{A_1  \eta}{\omega_2}  \sinh(\eta) 
\end{equation}
\begin{equation}\nonumber
 A_1 = \frac{\bar{s}_{HD}^{\ast}(z_D=1)}{\frac{\eta}{\omega_2}  \sinh(\eta) - \cosh(\eta)}
\end{equation}
which gives the final dimensionless Laplace-Hankel form of the saturated-zone solution as
\begin{equation}\nonumber
 \bar{s}_D^{\ast}(z_D)  = \bar{s}_{HD}^{\ast}(z_D) + \bar{s}_{HD}^{\ast}(z_D=1) \frac{\cosh(\eta z_D)}{\frac{\eta}{\omega_2}  \sinh(\eta) - \cosh(\eta)}
\end{equation} 
This expression is identical in form to \cite[eqn.\ C17]{mishra10},
with $\omega_2$ instead of $q$.  The final Laplace-Hankel form of the
unsaturated zone solution can be found by substituting $A_1$ into one
of the expressions for $\tilde\beta$ above, and substituting that into
\eqref{eq:sigma}.

\section{Formulation of Finite Difference Solution}
The ordinary differential equation in the vadose zone can be solved
via finite differences in space, while still within Laplace-Hankel
space.  This is done mostly as a check of the other derivation, and
partly becauase of inspiration derived from a pithy remark by a co-worker.

Substituting finite difference approximations for the $z_D$
derivatives in \eqref{eq:mn-d5-dimless} gives
\begin{equation}
  \nonumber
  \frac{1}{h^2} \left( \sigma_{j+1} - 2 \sigma_j + \sigma_{j+1}
  \right) - \frac{\beta_3}{h} \left( \sigma_{j+1} - \sigma_j\right) -
  \left( B_1 e^{-\beta_1 jh} + B_2 \right) \sigma_j
  = 0
\end{equation}
where the bar and star decorators on $\sigma$ are left out for simplicity, $h$ is
the dimensionless inter-node spacing (assumed constant), and
the subscript indicates the current node $j$ and the node one above
$j+1$ and below $j-1$.  The node indices run $0 \le j \le N-1$, where
 there are $N$ nodes and $h = L_D / (N - 1)$.

In the following $\omega_j = B_1 e^{-\beta_1 jh} + B_2$;
explicitly writing out the finite-difference matrix for a 3-node
problem with node 1 at $z_D=1$ and node 3 at $z_D=1+L/b$, without
explicitly handling the boundary conditions, is
\begin{equation}
  \label{eq:fd-01}
  \left[ \begin{matrix}
    \frac{1}{h^2} & \left(\frac{\beta_3}{h} - \frac{2}{h^2} - \omega_0\right) &
    \frac{1}{h^2} - \frac{\beta_3}{h} & 0 & 0 \\ 
    0 & \frac{1}{h^2} & \left(\frac{\beta_3}{h} - \frac{2}{h^2} - \omega_1 \right)&
    \frac{1}{h^2} - \frac{\beta_3}{h} & 0  \\ 
    0 & 0 & \frac{1}{h^2} & \left(\frac{\beta_3}{h} - \frac{2}{h^2}  -
      \omega_2 \right) &
    \frac{1}{h^2} - \frac{\beta_3}{h} \\ 
  \end{matrix}\right] 
\left[\begin{matrix}
\sigma_{-1} \\ \sigma_0 \\ \sigma_1 \\ \sigma_2 \\ \sigma_3
\end{matrix}\right]
=
\left[\begin{matrix}
0 \\ 0\\ 0 \\ 0 \\ 0
\end{matrix}\right]
\end{equation}
where $\sigma_{-1}$ and $\sigma_3$ are ``ghost'' nodes
which will be used to represent boundary conditions, but do not
physically exist; \eqref{eq:fd-01} is in the canonical form $\mathbf{Ax}=\mathbf{b}$.  Because
$\partial \sigma/\partial z =0$ at $z = b+L$, we can set $\sigma_2=
\sigma_3$, which means adding the two rightmost columns of
$\mathbf{A}$ and eliminating $\sigma_3$.

The node at the bottom of the vadose zone ($\sigma_0$) appears in two
boundary conditions, the head and flux continuity conditions with the
aquifer.  The flux continuity condition can be written in
finite-difference form as
\begin{equation}
 \nonumber
  \lim_{h \rightarrow 0} \frac{\sigma_0 - \sigma_{-1}}{h} = A_1 \eta \sinh(\eta)
\end{equation}
which can be used to eliminate $\sigma_{-1}$ from $\mathbf{A}$.
Substituting $\sigma_{-1} = \sigma_0 - hA_1 \eta \sinh(\eta)$ into
\eqref{eq:fd-01} gives the following equation for the first row of
$\mathbf{A}$
\begin{equation}
  \nonumber
  \frac{1}{h^2} \left[ \sigma_0 - hA_1 \eta \sinh(\eta)\right] +
  \left(\frac{\beta_3}{h} - \frac{2}{h^2}  - \omega_0\right) \sigma_0
  + \left(\frac{1}{h^2} - \frac{\beta_3}{h}\right) \sigma_1 = 0
\end{equation}
Further using head continuity equation $\sigma_0 = A_1 \cosh(\eta) +
s_{H}(z_D=1)$ to eliminate $\sigma_0$ from this first row leads to
\begin{equation}
 \nonumber
    \frac{A_1}{h}  \eta \sinh(\eta) +
  \left( \frac{\beta_3}{h} - \frac{1}{h^2} + - \omega_0\right) \left[
    A_1 \cosh(\eta) + s_{H}(1)\right] + \left(\frac{1}{h^2} -
    \frac{\beta_3}{h}\right) \sigma_1 = 0.
\end{equation}
Performing the substution for $\sigma_0$ into the second row of
$\mathbf{A}$, and regrouping to solve for $A_1$ in place of the
eliminated $\sigma_0$ leads to
 \begin{equation}
  \label{eq:fd-02}
  \left[ \begin{matrix}
     \cosh(\eta)c_1 - \frac{\eta}{h} \sinh(\eta) &
    \frac{1}{h^2} - \frac{\beta_3}{h} & 0 \\ 
     \frac{1}{h^2}\cosh(\eta) & c_2 &
    \frac{1}{h^2} - \frac{\beta_3}{h}   \\ 
     0 & \frac{1}{h^2} & c_2 +
    \frac{1}{h^2} - \frac{\beta_3}{h} \\ 
  \end{matrix}\right] 
\left[\begin{matrix}
A_1 \\ \sigma_1 \\ \sigma_2 
\end{matrix}\right]
=
\left[\begin{matrix}
-c_1 s_H(1) \\ 
-\frac{1}{h^2}s_H(1) \\ 0 
\end{matrix}\right]
\end{equation}
where $c_1 = \frac{\beta_3}{h} - \frac{1}{h^2} - \omega_j$ and $c_2 =
\frac{\beta_3}{h} - \frac{2}{h^2} - \omega_j $.  This 
$3\times 3$ case illustrates all the entries in a general $n\times
n$ problem; including boundary nodes and an interior node.  

The upper--left coefficient $\cosh(\eta)c_1 - \frac{\eta}{h}\sinh(\eta)$ in
\eqref{eq:fd-02} may be problematic due to cancellation between large
quantitites, so it can be re-written in terms of exponentials as 
\begin{equation}
 \nonumber
 \frac{1}{2} \left[ e^{\eta} \left( c_1 - \frac{\eta}{h}\right) + e^{-\eta}
   \left( c_1 + \frac{\eta}{h}\right) \right].
\end{equation}

The Thomas algorithm for tri-diagonal matrices can be used to
numerically solve for $A_1$ and $\sigma_j$, where $j=1,2, \dots, n$
(not including $\sigma_0$).  Once $A_1$ is found numerically, the
solution in the saturated zone is simply 
\begin{equation}
 \nonumber
 \bar{\sigma}^{\ast}_{D}(z_D) = \bar{\sigma}^{\ast}_{HD}(z_D) + A_1
 \cosh(\eta z_D).
\end{equation}
This involves computing the Thomas algorithm in Laplace-Hankel space,
for each combination of $(a,p)$.  The Thomas algorithm can be
vectorized to compute many solutions simulataneously.

For small problems ($n \le 5$) the finite difference matrix above can
be solved algebraically using Cramer's rule, or using equivalently
using Mathematica.  

\bibliographystyle{alpha}
\bibliography{mn}

\end{document}
